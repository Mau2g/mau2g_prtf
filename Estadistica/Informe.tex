\documentclass[12pt,a4paper]{report}

% Paquetes básicos
\usepackage[spanish]{babel}
\usepackage[utf8]{inputenc}
\usepackage[T1]{fontenc}
\usepackage{graphicx} % Para insertar imágenes
\usepackage{amsmath, amssymb}
\usepackage{booktabs} % Tablas bonitas
\usepackage{geometry}
\usepackage{hyperref}
\usepackage{caption}
\usepackage{float}
\usepackage{setspace}
\usepackage{longtable}
\geometry{margin=2.5cm}
\setstretch{1.5}

% Datos de portada
\title{Capacitación tecnológica como factor de eficiencia en el uso de sistemas de facturación electrónica en la fuerza laboral madura}
\author{Integrantes: \\
\\
Docente: }
\date{Ciudad -- Año}

\begin{document}

% ==================== CARÁTULA ====================
\begin{titlepage}
    \centering
    {\Large \textbf{Facultad de Ingeniería}}\\[0.5cm]
    {\large Carrera de Ingeniería ...}\\[3cm]
    {\LARGE \textbf{Informe de Proyecto}}\\[0.5cm]
    {\Large \textbf{Capacitación tecnológica como factor de eficiencia en el uso de sistemas de facturación electrónica en la fuerza laboral madura}}\\[3cm]
    \begin{flushleft}
        \textbf{Integrantes:} \\
        \vspace{1cm}
        \textbf{Docente:} \\
    \end{flushleft}
    \vfill
    {\large Ciudad -- Año}
\end{titlepage}

% ==================== ÍNDICE ====================
\tableofcontents
\newpage

% ==================== INTRODUCCIÓN ====================
\chapter{Introducción}

\section*{Contexto}
La empresa desarrolla soluciones tecnológicas para la gestión empresarial (facturación electrónica, POS, manejo de pedidos). En su expansión a distintos sectores, ha identificado que la fuerza laboral madura (mayores de 45--50 años) presenta mayores dificultades para adaptarse al uso de herramientas digitales.

\section*{Problema}
La brecha digital en trabajadores de mayor edad afecta la eficiencia en el uso de los sistemas:
\begin{itemize}
    \item Tiempos de facturación más largos.
    \item Mayores errores en registros.
    \item Mayor dependencia del soporte técnico.
    \item Impacto en la satisfacción de clientes.
\end{itemize}

\section*{Relevancia}
Este estudio busca evidenciar, mediante análisis estadístico, cómo la capacitación tecnológica puede mejorar la eficiencia de este grupo laboral y así optimizar la adopción de soluciones tecnológicas en distintos sectores.

\section*{Objetivos}
\textbf{General:} Evaluar cómo la capacitación tecnológica influye en la eficiencia del uso de sistemas de facturación electrónica en la fuerza laboral madura.

\textbf{Específicos:}
\begin{enumerate}
    \item Identificar el nivel de familiaridad tecnológica de los trabajadores maduros.
    \item Analizar indicadores de eficiencia (errores, tiempos de facturación, satisfacción).
    \item Estimar intervalos de confianza para los indicadores clave.
    \item Contrastar hipótesis sobre la relación entre capacitación y eficiencia.
\end{enumerate}

\begin{figure}[H]
    \centering
    \includegraphics[width=0.6\textwidth]{diagrama_conceptual.png}
    \caption{Diagrama conceptual: Edad → Capacitación → Eficiencia tecnológica.}
\end{figure}

% ==================== PLANTEAMIENTO DEL PROBLEMA ====================
\chapter{Planteamiento del problema e hipótesis}

\section*{Problema específico}
¿La capacitación tecnológica mejora la eficiencia en el uso de sistemas de facturación electrónica en la fuerza laboral madura?

\section*{Hipótesis}
\begin{itemize}
    \item $H_{0}$: La capacitación tecnológica no influye en la eficiencia de la fuerza laboral madura en el uso de sistemas de facturación electrónica.
    \item $H_{1}$: La capacitación tecnológica sí influye significativamente en la eficiencia de la fuerza laboral madura en el uso de sistemas de facturación electrónica.
\end{itemize}

\section*{Variables}
\begin{longtable}{p{4cm}p{5cm}p{4cm}}
\toprule
\textbf{Variable} & \textbf{Definición} & \textbf{Tipo / Medición} \\
\midrule
Independiente & Nivel de capacitación tecnológica (horas de capacitación, evaluaciones) & Cuantitativa / Encuestas, registros \\
Dependiente 1 & Tiempo de facturación & Cuantitativa / Segundos por transacción \\
Dependiente 2 & Número de errores & Cuantitativa / Conteo de errores \\
Dependiente 3 & Satisfacción de clientes & Cualitativa / Escala Likert \\
\bottomrule
\end{longtable}

% ==================== METODOLOGÍA ====================
\chapter{Metodología}

\section*{Tipo de estudio}
Cuantitativo, descriptivo e inferencial.

\section*{Datos}
\begin{itemize}
    \item Encuestas y cuestionarios al personal (nivel de confianza tecnológica, autopercepción de habilidades).
    \item Datos del sistema (tiempo promedio de facturación, errores por transacción).
    \item Encuestas de satisfacción de clientes internos/externos.
\end{itemize}

\section*{Técnicas estadísticas}
\begin{itemize}
    \item Descriptiva: medias, desviación estándar, porcentajes.
    \item Distribuciones muestrales: medias de tiempos de facturación, proporción de errores.
    \item Intervalos de confianza: para tiempos y errores.
\end{itemize}

\section*{Justificación}
Estas técnicas permiten comprobar de forma objetiva si la capacitación tiene un efecto significativo en la eficiencia.

\begin{figure}[H]
    \centering
    \includegraphics[width=0.8\textwidth]{metodologia.png}
    \caption{Diagrama de flujo de la metodología.}
\end{figure}

% ==================== ANÁLISIS DE DATOS ====================
\chapter{Análisis de datos}
\begin{itemize}
    \item Edad promedio del personal.
    \item Horas promedio de capacitación recibida.
    \item Tiempo promedio de facturación por grupo (con y sin capacitación).
    \item Porcentaje de errores cometidos.
\end{itemize}

% Ejemplo de gráficos
\begin{figure}[H]
    \centering
    \includegraphics[width=0.7\textwidth]{histograma_edades.png}
    \caption{Histograma de distribución de edades.}
\end{figure}

\begin{figure}[H]
    \centering
    \includegraphics[width=0.7\textwidth]{errores_capacitacion.png}
    \caption{Errores con vs. sin capacitación.}
\end{figure}

% ==================== DISTRIBUCIONES MUESTRALES ====================
\chapter{Cálculo de distribuciones muestrales}

Ejemplo: distribución muestral de la media del tiempo de facturación en personal mayor de 50 años. Incluir fórmulas, supuestos y cálculo paso a paso.

% ==================== INTERVALOS DE CONFIANZA ====================
\chapter{Intervalos de confianza}

Se calcularán intervalos para:
\begin{itemize}
    \item Tiempo promedio de facturación.
    \item Porcentaje de errores.
    \item Nivel de satisfacción.
\end{itemize}

\begin{figure}[H]
    \centering
    \includegraphics[width=0.7\textwidth]{intervalos_confianza.png}
    \caption{Intervalos de confianza para indicadores clave.}
\end{figure}

% ==================== INTERPRETACIÓN ====================
\chapter{Interpretación de resultados}

Analizar si los intervalos confirman la hipótesis. Interpretar la relación entre edad, capacitación y eficiencia.

\begin{itemize}
    \item Mejora en eficiencia operativa.
    \item Mayor satisfacción de clientes.
    \item Facilidad de expansión a nuevos sectores.
\end{itemize}

% ==================== CONCLUSIONES ====================
\chapter{Conclusiones y recomendaciones}

\section*{Conclusiones}
Resumir hallazgos clave y confirmar o rechazar la hipótesis.

\section*{Recomendaciones}
\begin{itemize}
    \item Diseñar programas de capacitación dirigidos a trabajadores maduros.
    \item Implementar evaluaciones periódicas.
    \item Invertir en usabilidad de sistemas tecnológicos adaptados a distintas edades.
\end{itemize}

\begin{figure}[H]
    \centering
    \includegraphics[width=0.7\textwidth]{antes_despues.png}
    \caption{Comparativo antes y después de la capacitación.}
\end{figure}

% ==================== ANEXOS ====================
\chapter{Anexos}
\begin{itemize}
    \item Bases de datos de ejemplo (encuestas, registros).
    \item Fórmulas detalladas.
    \item Gráficos adicionales.
\end{itemize}

% ==================== REFERENCIAS ====================
\chapter{Referencias}
\bibliographystyle{apalike}
\begin{thebibliography}{9}

\bibitem{inei}
INEI (2023). \textit{Estadísticas de la población laboral en el Perú}.

\bibitem{mintra}
MINTRA (2022). \textit{Informe sobre brecha digital y empleo en el Perú}.

\bibitem{produce}
Ministerio de la Producción (2023). \textit{Adopción tecnológica en empresas peruanas}.

\bibitem{sunat}
SUNAT (2021). \textit{Normativa sobre facturación electrónica en el Perú}.

\bibitem{scielo}
Autores varios. Artículos académicos en Scielo y Redalyc sobre capacitación digital en adultos.

\end{thebibliography}

\end{document}
